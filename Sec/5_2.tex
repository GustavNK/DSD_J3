\documentclass[../journal.tex]{subfiles}
\graphicspath{{Figs/5_2/}{../Figs/5_2/}}  %Path from main  and path from this file til graphics

% Source input
% \begin{table}[H]
%     \centering
%       \framebox{
%         \rule{8pt}{0pt}
%           \lstinputlisting[firstline=1,lastline=16]{../src/code_name.vhd}
%   }
%   \caption{Kode for 4bit-adder med unsigned}	
%   \label{src:Tab11}
% \end{table}

% Pictures{name of file}{size compared to page}{captions}{label}
% \pic{Unsigned1.png}{1}{opbygning af 4bit-adder}{Fig11}



%-----------DOCUMENT--------------

\begin{document}

\subsection{Introduktion}
Formålet med denne del af øvelsen vil være at konstruere et `guess game' der evaluere et input der indtastes af en `spiller' som enten `Hi' `Lo' eller `-', og fortæller om du har gættet rigtigt på den hemmelig værdi indtastet af en anden `spiller'.

\subsection{Design og implementering}
Til denne øvelse benyttes der en \textit{structural} struktur. Systemmet består af to(2) Muxes, to(2) bin2hex som vi byggede i 5.1 og et `Compare logic' modul, til at styre fremvisningen af `Hi', `Lo' eller `-'. Modsat vores tidligere arbejde benytter vi også her en latch, idet specifikt har brug for at huske en værdi, her vores gæt-input, til evaluering. Nedenfor kan koden ses for vores `Compare logic' og Latch. Noter at vi ikke har indkluderet koden for multiplexerne, idet dette virker trivielt, og er blevet vist i en tidligere opgave:

\begin{table}[H]
    \centering
      \framebox{
        \rule{8pt}{0pt}
          \lstinputlisting{../src/5/mylatch.vhd}
  }
  \caption{mylatch.vhd}	
  \label{src:mylatch}
\end{table}

\begin{table}[H]
    \centering
      \framebox{
        \rule{8pt}{0pt}
          \lstinputlisting{../src/5/compare_logic.vhd}
  }
  \caption{compare\_logic.vhd}
  \label{src:compare_logic}
\end{table}

`Structural' strukturen benyttes nu til at sammensætte vores `Guess game' fra de tidligere beskrevne kode-blokke. Koden for Guess Game, samt koden for den tilhørende tester kan findes nedenfor:

\begin{table}[H]
    \centering
      \framebox{
        \rule{8pt}{0pt}
          \lstinputlisting{../src/5/guess_game.vhd}
  }
  \caption{guess\_game.vhd}	
  \label{src:guess_game}
\end{table}

\begin{table}[H]
  \centering
    \framebox{
      \rule{8pt}{0pt}
        \lstinputlisting{../src/5/guess_game_test.vhd}
}
\caption{guess\_game\_test.vhd}	
\label{src:guess_game_test}
\end{table}

Til sidst foretog vi en realisering af vores system på DE2-Board. Et gæt-input på værdien XX, med et output på `Hi', der indikere et rigtigt gæt, kan ses på Fig.%\ref{fig:5_2 realisering}:

\pic{5_2}{0.7}{Realisering af bin2hex}{fig}


\subsection{Resultater og diskussion}

Fra vores realisering så vi den ønskede opførsel af vores Guess Game. Den kunne tage alle de ønskede inputs, og korrekt fremvise hvorvidt de var det korrekte svar.\newline \newline
Hvis vi kigger på hvordan Quartus har konstrueret systemmet ud fra vores kode, ser vi brug af latches i voes mylatch.vhd. Dette var selvfølgelig tiltænkt, og viser blot at vi har formået korrekt at implementere vores latch.


\subsection{Konklusion}


\end{document}