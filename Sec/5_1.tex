\documentclass[../journal.tex]{subfiles}
\graphicspath{{Figs/5_1/}{../Figs/5_1/}}  %Path from main  and path from this file til graphics

% --Source input--
% \begin{table}[H]
%     \centering
%       \framebox{
%         \rule{8pt}{0pt}
%           \lstinputlisting[firstline=1,lastline=16]{../src/code_name.vhd}
%   }
%   \caption{Kode for 4bit-adder med unsigned}	
%   \label{src:Tab11}
% \end{table}

% Pictures{name of file}{size compared to page}{captions}{label}
% \pic{Unsigned1.png}{1}{opbygning af 4bit-adder}{Fig11}

%-----------DOCUMENT--------------

\begin{document}

\subsection{Introduktion}
I denne del af øvelse 5 vil vi implementere en 7-segment LED-display som vil tage binære tal som input. Som krav til strukturen skal der benyttes case-statemens.

\subsection{Design og implementering}
Som nævnt i introduktonen er det et krav at case-statements skal benyttes til strukturen af koden. Vores 7-segment display skal kunne fremvise hex-tallene fra 1 til 15, altså fra 1 til F. Vi opstiller derfor when-statements for alle 15 forventede binære input, der vil sende binære-kombinationer til 7-segment displayet så de ønskde hex-tal fremvises. Idet 7-segment displayet er active-LOW har vi dog inverteret de de binære-kombinationer. Der opstilles desuden et when-statement for alle udenligende muligheder, der blot vil give et blankt display. Koden kan ses nedenfor:

\begin{table}[H]
    \centering
      \framebox{
        \rule{8pt}{0pt}
          \lstinputlisting{../src/bin2hex.vhd}
  }
  \caption{Kode for bin2hex.vhd}	
\end{table}

Vi har desuden lavet en tester for forrige kode, idet vi gerne vil foretage en realisering af koden på DE2-Boardet. Koden for testeren kan ses nedenfor:

\begin{table}[H]
    \centering
      \framebox{
        \rule{8pt}{0pt}
          \lstinputlisting{../src/bin2hex_test.vhd}
  }
  \caption{Kode for bin2hex\_test.vhd}	
\end{table}

%\begin{table}[H]
%    \centering
%      \framebox{
%        \rule{8pt}{0pt}
%          \lstinputlisting[firstline=1,lastline=9]{../src/5/bin2hex.vhd}
%  }
%  \caption{Entity for bin2hex.vhd}	
%  \label{src:bin2hex_ent}
%\end{table}

%s
%\begin{table}[H]
%    \centering
%        \framebox{
%        \rule{8pt}{0pt}
%            \lstinputlisting[firstline=11,lastline=35]
%            {../src/5/bin2hex.vhd}
%    }
%    \caption{Architechture for bin2hex.vhd}	
%    \label{src:bin2hex_arch}
%\end{table}


%\begin{table}[H]
%    \centering
%        \framebox{
%        \rule{8pt}{0pt}
%            \lstinputlisting
%            {../src/5/bin2hex_test.vhd}
%    }
%    \caption{Design af testbench for bin2hex}	
%    \label{src:bin2hex_test}
%\end{table}

\subsection{Resultater og diskussion}


Fra vores realisering af koden så vi den forventede opførsel af en binær-til-7-segment konverter. Vi kunne give binære tal fra 1 til 15 som input, og se de tilsvarende hextal på vores 7-segment display. Input af det binære tal 12 på switchches, med et output af hextallet `C' kan ses på Fig.\ref{fig:bin2hex} :

\pic{rea1}{0.7}{Realisering af bin2hex}{fig:bin2hex}

Hvis vi kigger på RTL-viewer diagrammet of vores struktur, kan ses på Fig.\ref{fig:rtl_bin2hex1}, ser vi at den ikke indeholder nogen latches. Dette skyldes formodentligt vores indkludering af et \textit{Others}-statement, der tager forhold for alle de mulige inputs vi ikke specifikt har designeret gennem et \textit{when}-statement. Derfor opbygges struktur blot gennem Muxes.

\pic{rtl_case.png}{0.4}{RTL view for bin2hex}{fig:rtl_bin2hex1}

\subsection{Konklusion}

Vi konkluderer at vi har produceret en binært-til-7-segment-decoder på et DE2-Board, der kan fremvise fra 1 til F i hextal. Systemmet opbygges gennem Muxes, og indeholder derfor ingen latches, som ønsket. 

\end{document}