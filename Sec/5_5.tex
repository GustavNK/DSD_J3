\documentclass[../journal.tex]{subfiles}
\graphicspath{{Figs/5_5/}{../Figs/5_5/}}  %Path from main  and path from this file til graphics

% Source input
% \begin{table}[H]
%     \centering
%       \framebox{
%         \rule{8pt}{0pt}
%           \lstinputlisting[firstline=1,lastline=16]{../src/code_name.vhd}
%   }
%   \caption{Kode for 4bit-adder med unsigned}	
%   \label{src:Tab11}
% \end{table}

% Pictures{name of file}{size compared to page}{captions}{label}
% \pic{Unsigned1.png}{1}{opbygning af 4bit-adder}{Fig11}



%-----------DOCUMENT--------------

\begin{document}

\subsection{Introduktion}
I denne af øvelsen vil vi benytte et loop-statement til at læse en logisk tælning op i et array, og outputte det på et 7-segment display.

\subsection{Design og implementering}
Vores design er fokuseret på at igen benytte loop-statements. Variablen \textit{count} sættes først lig 0. Derfefter køres alle elementerne i arrayet `A' igennem vha. et loop-statement. Hvis et element er lig `1', vil count variablen tælles en(1) op. Når alle elementer i `A' er kørt igennem laves count først til \textit{integer}, derefter \text{std\_logic\_vector} for at kunne lægges over i output vektoren ones. Koden for tælleren samt koden for testeren til brug på DE2-Boardet kan findes nedenfor:

\begin{table}[H]
    \centering
      \framebox{
        \rule{8pt}{0pt}
          \lstinputlisting[firstline=1,lastline=30]{../src/5/counting_ones.vhd}
  }
  \caption{counting\_ones.vhd}	
  \label{src:TabConTest}
\end{table}

\begin{table}[H]
    \centering
      \framebox{
        \rule{8pt}{0pt}
          \lstinputlisting[firstline=1,lastline=30]{../src/5/count_ones_test.vhd}
  }
  \caption{count\_ones\_test.vhd}	
  \label{src:TabConTest}
\end{table}

%\pic{rtl_case.png}{0.4}{RTL view for bin2hex}{fig:rtl_bin2hex}

Til sidst foretog vi en realisering på DE2-Boardet vha. vores tester-fil. 


\subsection{Resultater og diskussion}




\subsection{Konklusion}

\end{document}